\section{はじめに}
\hspace{\parindent}
本共同研究では,これまで粗視化分子動力学法(CGMD法)のプログラムを開発し,
粘土含水系の組織構造形成に関するシミュレーションを行ってきた.
昨年度の研究では,粘土含水系が環境(外界)との間で水分をやり取りし,
吸水や排水が生じることを模擬シミュレーションを行うためのメカニズムとして,
化学ポテンシャル一定条件での計算が可能な形へプログラムを拡張した.
化学ポテンシャルはその逆数が湿度の高低に相当するパラメータとしての物理的意味を持つ.
従って,設定した化学ポテンシャルが高い(環境の湿度が低い)ときに排水を,
低い(環境の湿度が高い)ときには吸水を起こす方向へ粘土含水系の状態は変化する.
その結果,体積が拘束されている場合は吸水(排水)によっって膨潤圧力が上昇(低下)する.
体積が拘束されていない場合は吸水によって体積膨張を,排水によって体積収縮を生じる.
%
このような吸排水の原動力は粘土分子表面への水和に関するエネルギーにある.
水分子は電荷を帯びた粘土表面に水和することでよりエネルギーの低い安定な状態になる.
水和に起因したこのエネルギー変化を水和エネルギーと呼ぶ.
CGMD法では水和エネルギーと水和水量の関係を仮定し,モンテカルロ法
よる緩和計算を経て系の平衡状態における水分量と配置を決定することができる.
%
この方法により昨年度の研究では,水和エネルギーが水和量の増加に対して
単調減少すると仮定して計算を行った.その結果,吸排水や膨潤厚が,
化学ポテンシャルに応じて期待したように生じることが確認できた.
しかしながら,単調減少する水和エネルギー関数は,一つのパラメータで
規定される単純なもので,層間イオン種に応じたモンモリロナイトの複雑な
膨潤挙動を再現するためには十分でない.
例えばNa型モンモリロナイトでは,相対湿度に対して,ステップ状の膨潤を
示すことがX線観測の結果として知られている.
このように離散的な膨潤状態を取る挙動は,単調な水和エネルギー関数からは生じ得ず,
モンモリロナイトの膨潤挙動を表現するためには,いくつかの極値をもつような非単調な
水和エネルギーモデルが必要であることを意味する.そのような水和エネルギーモデル
を開発するには,水和エネルギーの局所的な変動が,膨潤挙動にどのような影響を与える
かについて十分理解することが必要となる.そこで本年度は,水分量に対して単調に増加する
昨年度までのモデルに振動成分を加え,水和エネルギーの局所的変動が膨潤に与える
影響を調べる.その結果,新しい水和エネルギーモデルを用いた計算では,
水和エネルギーの極大点を避けるように膨潤状態が決まり,
化学ポテンシャルの変化に対して極大点付近では膨潤状態が離散的に推移することをを示す.


以下では,はじめにCGMD法の基本的な考え方を始めに述べる.
特にCGMDモデルにおける水和水量の表現について要点を述べる.
次に,今回の解析に用いた水和エネルギーモデルの詳細とその意図を説明する.
続いて,温度,体積,化学ポテンシャル一定の条件で行った緩和
シミュレーションの結果を示す.その際,どのような膨潤状態が
支配的であるかを見るために,水和水層厚の頻度分布を示す.
この頻度分布から膨潤状態が化学ポテンシャルに対して必ずしも連続的に変化せず,
水和エネルギーの極大点を避けるように水和状態が選択されることを示す.
最後に,本年度の研究結果についてまとめ今後の課題を示す.
%	CGMD法の将来性,利用方法について
%なお,今回の研究で得られた成果を踏まえれば,今後,水和エネルギーモデルをより精緻化することで
%化学ポテンシャルを変化させながら膨潤量を計算することや, 所定の化学ポテンシャルにおいて生じる
%膨潤圧の計算が可能となる.これらの値は実測値と比較することができ,例えば,
%X線回折試験で得られた膨潤曲線や膨潤圧の計測値とシミュレーション結果を比較すれば,
%層間イオンの種類や組成に応じた膨潤挙動の解釈や推定にも利用できると期待される.
%CGMD法では,このような膨潤解析を任意の温度や乾燥密度,飽和度で行うことができるため,
%その信頼性が担保されれば,実験が困難な温度や密度,湿度条件での膨潤圧や膨潤および水分量
%を推定することにも役立つ.このようなシミュレーションが実現すれば,ベントナイト緩衝材の
%再冠水時の浸透や膨潤挙動を調べる上で有用な知見与えることが期待できる.
