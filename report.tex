\documentclass[11pt,a4j]{jarticle}
\usepackage[dvipdfmx]{graphicx,color}
%\usepackage{showkeys}
\usepackage{wrapfig}
\usepackage{amssymb}
\setlength{\topmargin}{-1.5cm}
%\setlength{\textwidth}{15.5cm}
\setlength{\textheight}{25.2cm}
\newlength{\minitwocolumn}
\setlength{\minitwocolumn}{0.5\textwidth}
\addtolength{\minitwocolumn}{-\columnsep}
%\addtolength{\baselineskip}{-0.1\baselineskip}
%
\def\Mmaru#1{{\ooalign{\hfil#1\/\hfil\crcr
\raise.167ex\hbox{\mathhexbox 20D}}}}
%
\begin{document}
\newcommand{\fat}[1]{\mbox{\boldmath $#1$}}
\newcommand{\D}{\partial}
\newcommand{\w}{\omega}
\newcommand{\ga}{\alpha}
\newcommand{\gb}{\beta}
\newcommand{\gx}{\xi}
\newcommand{\gz}{\zeta}
\newcommand{\vhat}[1]{\hat{\fat{#1}}}
\newcommand{\spc}{\vspace{0.7\baselineskip}}
\newcommand{\halfspc}{\vspace{0.3\baselineskip}}
\bibliographystyle{unsrt}
%\pagestyle{empty}
\newcommand{\twofig}[2]
 {
   \begin{figure}[h]
     \begin{minipage}[t]{\minitwocolumn}
         \begin{center}   #1
         \end{center}
     \end{minipage}
         \hspace{\columnsep}
     \begin{minipage}[t]{\minitwocolumn}
         \begin{center} #2
         \end{center}
     \end{minipage}
   \end{figure}
 }
%%%%%%%%%%%%%%%%%%%%%%%%%%%%%%%%%
%\vspace*{\baselineskip}
\begin{center}
{\Large \bf 令和3年度 共同研究報告書}
\end{center}
\vspace{2mm}
\begin{center}
{\LARGE \bf 
メソスケールシミュレーションによる\\緩衝材の特性評価に関する研究} 
\end{center}
\begin{center}
岡山大学学術研究院 環境生命科学学域\\
木本和志
\end{center}
\vspace{10mm}
%%%%%%%%%%%%%%%%%%%%%%%%%%%%%%%%%%%%%%%%%%%%%%%%%%%%%%%%%%%%%%%%
\section{はじめに}
\hspace{\parindent}
本共同研究では,これまで粗視化分子動力学法(CGMD法)のプログラムを開発し,
粘土含水系の組織構造形成に関するシミュレーションを行ってきた.
昨年度の研究では,粘土含水系が環境(外界)との間で水分をやり取りし,
吸水や排水が生じることを模擬シミュレーションを行うためのメカニズムとして,
化学ポテンシャル一定条件での計算が可能な形へプログラムを拡張した.
化学ポテンシャルはその逆数が湿度の高低に相当するパラメータとしての物理的意味を持つ.
従って,設定した化学ポテンシャルが高い(環境の湿度が低い)ときに排水を,
低い(環境の湿度が高い)ときには吸水を起こす方向へ粘土含水系の状態は変化する.
その結果,体積が拘束されている場合は吸水(排水)によっって膨潤圧力が上昇(低下)する.
体積が拘束されていない場合は吸水によって体積膨張を,排水によって体積収縮を生じる.
%
このような吸排水の原動力は粘土分子表面への水和に関するエネルギーにある.
水分子は電荷を帯びた粘土表面に水和することでよりエネルギーの低い安定な状態になる.
水和に起因したこのエネルギー変化を水和エネルギーと呼ぶ.
CGMD法では水和エネルギーと水和水量の関係を仮定し,モンテカルロ法
よる緩和計算を経て系の平衡状態における水分量と配置を決定することができる.
%
この方法により昨年度の研究では,水和エネルギーが水和量の増加に対して
単調減少すると仮定して計算を行った.その結果,吸排水や膨潤厚が,
化学ポテンシャルに応じて期待したように生じることが確認できた.
しかしながら,単調減少する水和エネルギー関数は,一つのパラメータで
規定される単純なもので,層間イオン種に応じたモンモリロナイトの複雑な
膨潤挙動を再現するためには十分でない.
例えばNa型モンモリロナイトでは,相対湿度に対して,ステップ状の膨潤を
示すことがX線観測の結果として知られている.
このように離散的な膨潤状態を取る挙動は,単調な水和エネルギー関数からは生じ得ず,
モンモリロナイトの膨潤挙動を表現するためには,いくつかの極値をもつような非単調な
水和エネルギーモデルが必要であることを意味する.そのような水和エネルギーモデル
を開発するには,水和エネルギーの局所的な変動が,膨潤挙動にどのような影響を与える
かについて十分理解することが必要となる.そこで本年度は,水分量に対して単調に増加する
昨年度までのモデルに振動成分を加え,水和エネルギーの局所的変動が膨潤に与える
影響を調べる.その結果,新しい水和エネルギーモデルを用いた計算では,
水和エネルギーの極大点を避けるように膨潤状態が決まり,
化学ポテンシャルの変化に対して極大点付近では膨潤状態が離散的に推移することをを示す.


以下では,はじめにCGMD法の基本的な考え方を始めに述べる.
特にCGMDモデルにおける水和水量の表現について要点を述べる.
次に,今回の解析に用いた水和エネルギーモデルの詳細とその意図を説明する.
続いて,温度,体積,化学ポテンシャル一定の条件で行った緩和
シミュレーションの結果を示す.その際,どのような膨潤状態が
支配的であるかを見るために,水和水層厚の頻度分布を示す.
この頻度分布から膨潤状態が化学ポテンシャルに対して必ずしも連続的に変化せず,
水和エネルギーの極大点を避けるように水和状態が選択されることを示す.
最後に,本年度の研究結果についてまとめ今後の課題を示す.
%	CGMD法の将来性,利用方法について
%なお,今回の研究で得られた成果を踏まえれば,今後,水和エネルギーモデルをより精緻化することで
%化学ポテンシャルを変化させながら膨潤量を計算することや, 所定の化学ポテンシャルにおいて生じる
%膨潤圧の計算が可能となる.これらの値は実測値と比較することができ,例えば,
%X線回折試験で得られた膨潤曲線や膨潤圧の計測値とシミュレーション結果を比較すれば,
%層間イオンの種類や組成に応じた膨潤挙動の解釈や推定にも利用できると期待される.
%CGMD法では,このような膨潤解析を任意の温度や乾燥密度,飽和度で行うことができるため,
%その信頼性が担保されれば,実験が困難な温度や密度,湿度条件での膨潤圧や膨潤および水分量
%を推定することにも役立つ.このようなシミュレーションが実現すれば,ベントナイト緩衝材の
%再冠水時の浸透や膨潤挙動を調べる上で有用な知見与えることが期待できる.
	
\section{粗視化分子動力学法(CGMD法)の概要}
\subsection{粒子間相互作用力}
CGMD法では,粗視化粒子に作用する力として分子内力と分子間力の2つを与える.
分子内力は,粗視化粒子を結合して一つの分子として振る舞うように剛性を与える
もので,本CGMD法では同一分子内で隣接する粗視化粒子を線形バネで結合すること
によって与えている.そのため,無応力状態では一つの分子を表現する粗視化粒子群
は直線上に整列する.
分子間力はファンデルワールス力の基本的なモデルであるレナード-ジョーンズ対ポテンシャル
(以下LJポテンシャル呼ぶ):
\begin{equation}
	U(\fat{x}_i,\fat{x}_j; \sigma) 
	= 4 \varepsilon 
	\left\{ 
	\left(\frac{\sigma}{r_{ij}}\right)^{12}
	-
	\left(\frac{\sigma}{r_{ij}}\right)^6
	\right\}, \ \ \left( r_{ij}=\left| \fat{x}_i-\fat{x}_j\right| \right)
	\label{eqn:LJ}
\end{equation}
で与える.
ここで,$\fat{x}_i$と$\fat{x}_j$は,それぞれ第$i$および第$j$番目の粗視化粒子の位置を,
$\varepsilon$と$\sigma$は,LJポテンシャルのパラメータである.$\varepsilon$は
\begin{equation}
	\varepsilon=1.0\times 10^{-19}, \ \ [{\rm Nm}]
	\label{eqn:eps_of_LJ}
\end{equation}
であり,$\sigma$は以下に述べるように粗視化粒子に水和した水分量を表す変数として用いる.

LJポテンシャルは粒子間距離$r_{ij}$が小さいときに斥力を,大きいときに引力を粒子間に作用させる.
引力と斥力は概ね$r_{ij}=1.1\sigma$を境に切り替わる.
斥力が作用する領域では$r_{ij}\rightarrow 0$に向けて斥力が急増する.
従って,$r_{ij}=\sigma$は粗視化粒子の接近限界とみなすことができる.
水を非圧縮性物質とみなせば,粗視化粒子の接近限界は粘土分子が当該位置でもつ
水和水量できまる.いま,粒子$i$のもつ水和水の層厚が$s_i$,
粒子$j$の水和水層厚が$s_j$であったとする.無水状態での粘土分子層の厚さを$\sigma_0$とすれば,
粗視化粒子の接近限界$\sigma$は次の式で与えられる.
\begin{equation}
	\sigma=\sigma_0 +s_i+s_j
	\label{eqn:}
\end{equation}
このことより,粗視化粒子の有する水和水層の厚さ$s_i$を定めればLJポテンシャルの
特性距離$\sigma$が決まり,水和水の量や分布に応じた粒子間相互作用を与えることができる.
CGMD法では,各粗視化粒子は位置$\fat{x}$と速度$\fat{v}$,向き$\fat{n}$を属性として持つ.
ここで,粘土分子の一方の面を正方向$\fat{n}$とし,その反対の方向を$-\fat{n}$とすれば,
これら2つの方向それぞれに水和水が存在するため,$\pm\fat{n}$方向の水和水層厚を
区別する場合$s^\pm$, 文脈から明らかな場合は$\pm$を省略して$s_i$などと書く.
ここい$i$は粗視化粒子番号を意味する.粒子番号についても明示する必要の無い場合は省略し
単に$s$と書く.これら$\fat{x},\fat{v},\fat{n}$および$s$は全て未知量である.
これらのうち$\fat{x}$と$\fat{v}$は,運動方程式:
\begin{equation}
	m\dot{ \fat{v}}=\fat{F}_K+\fat{F}_{LJ}, \ \ \fat{v}=\dot{\fat{x}}
	\label{eqn:eq_motion}
\end{equation}
を時間積分することで与えられた初期状態繰り返し状態を更新する.
なお,$m$は粗視化粒子の質量を, $\fat{F}_K$と$\fat{F}_{LJ}$は,
それぞれ分子内および分子間力を表す.粒子の向き$\fat{n}$は,分子を構成する
粗視化粒子の位置座標から各粒子位置での接ベクトルを求め,それに直交する方向として定める.
水和水層の厚さ$s$は,後に述べるように水和エネルギーと粒子間相互作用エネルギー
の停留条件から決定する.水和水層の厚さ$s$とX線回折試験から求められている
膨潤状態および層間距離は表\ref{tbl:tbl_sig}の通りである.
\begin{table}[h]
	\begin{center}
	\caption{分子間相互作用ポテンシャルにおける特性距離と膨潤状態の対応.}
	\vspace{3mm}
	\begin{tabular}{c||c|c|c|c|c}
		膨潤状態 & 0層 & 1層 & 2層 & 3層 & $\cdots$\\
		\hline
		特性距離(層間距離)$\sigma$[{\rm nm}]& 0.9 & 1.2 & 1.5 & 1.8 & $\cdots$ \\
		\hline
		水和水層厚$s$[{\rm nm}] & 0.0 & 0.15 & 0.30 & 0.45 & $\cdots$
	\end{tabular}
	\label{tbl:tbl_sig}
	\end{center}
\end{table}
%--------------------
\subsection{水分に関係したエネルギー}
粘土含水系を表現する粗視化粒子系は,水分の配置や量に応じて系全体がもつエネルギーを変化させる.
水分の量や配置に依存するエネルギーには,次のものが考えられる.
\begin{itemize}
\item
	粗視化粒子間相互作用のエネルギー(LJポテンシャル)$U_{LJ}$
\item
	水和エネルギー$U_{hyd}$
\item
	系内の水分量に依存するエネルギー$U_N$
\end{itemize}
粒子間相互作用エネルギー$U_{LJ}$は,LJポテンシャル(\ref{eqn:LJ})の特性距離$\sigma$を
通じて水分位置と量の影響を受ける.上記$U_{LJ}$は粒子径の全ポテンシャルエネルギーである
ため,CGMDを法では次の式で与えられる.
\begin{equation}
	U_{LJ}=\sum_{i \neq j} U(\fat{x}_i,\fat{x}_j; \sigma) 
	\label{eqn:}
\end{equation}
$U_{hyd}$は,水分が粘土分子の表面や層間に水和することによるエネルギー変化を表す.
水和水層の厚さが$s$における粗視化粒子の水和エネルギーを$u(s)$とすれば,
$U_{hyd}$は
\begin{equation}
	U_{hyd} =\sum_{\alpha=\pm} \sum_{i} u(s_i^{\pm})
	\label{eqn:Uhyd_tot}
\end{equation}
と表される.なお,ここでは水和エネルギーの基準を無水状態にとり,$u(0)=0$としている.
モンモリロナイトは大きなは膨潤や膨潤圧を生じることからも明らかなように,
湿潤環境下で強い水和が生じる.従って,水和エネルギー$U_{hyd}$は$U_{LJ}$との比較において
無視できない.
最後に,$U_{N}$は系内に存在する水分量の変動だけに起因したエネルギーを表す.
一般に,系外から物質を持ち込み,系の組成を変化させるためにはエネルギーが必要とされ,
$U_N$は系内外での水分の授受に関するものである.
水分粘土含水系と環境をひとまとめにして見たとき,粘土含水系内の水分と環境中の水分では
濃度が異なる.そのため粘土含水系は,系内での水分位置によらず,系外との濃度差に
起因したエネルギー$U_N$を持つ.
CGMD法では水分を粗視化粒子の属性として表現するため,水分や水分子間の相互作用力や
相互作用エネルギーは計算されない.従って,これら$U_{hyd}$や$U_{N}$はモデルや
計算条件として別途与える必要がある.

熱力学の第二法則に従えば,水分量と位置の自発的変化はこれら3つのエネルギーの和
\begin{equation}
	E=U_{LJ}+U_{hyd}+U_{N}
	\label{eqn:}
\end{equation}
が減少する方向に起きる.このことは,水分位置と量の変化に関する全微分を使って
\begin{equation}
	dE = dU_{LJ}+dU_{hyd}+dU_N <0
	\label{eqn:dE}
\end{equation}
と表され,平衡状態では停留条件$dE=0$満足する.なお,$U_N$は水分配置に依らないため,
系内の水分子数を$N$とすれば,
\begin{equation}
	dU_N=\mu dN, \ \ \left(  \mu:=\frac{\partial U_N}{\partial N}\right)
	\label{eqn:}
\end{equation}
とできる.ここでh$\mu=\frac{\partial U_N}{\partial N}$は水分に関する化学ポテンシャルを表す.
化学ポテンシャル一定のもとで計算を行う場合,式(\ref{eqn:dE})右辺の各項を計算し,
$dE<0$となる方向に水分状態を推移させる.この作業を繰り返し,最終的に停留条件$dE=0$
が満足される状態まで粗視化粒子径を緩和させれば,平衡状態における水分量と配置が決定できる.
その際,指定された化学ポテンシャルが$\mu>0$であれば,系内の水分が増加するとき$dN>0$で$U_N$も増加する.
これに対し水和エネルギー$U_{hyd}$は,水分量に関する負の減少関数のため,$dN>0$のとき$dU_{hyd}<0$である.
一方,$dU_{LJ}$は水分量の増加に対して正にも負にも成り得る.例えば,周囲の粒子から斥力を受ける
粗視化粒子に水分が追加されると,斥力はさらに大きくなりポテンシャル値も増加する.
逆に,粗視化粒子が疎に配置された状態で水分が追加されれば,粒子間の引力が増しポテンシャルは減少する.
このように,水和エネルギー$U_{hyd}$と水分量に関する$U_N$は相反する応答を示し
,ポテンシャルエネルギー$U_{LJ}$は粒子配置によって正負どちらの方向にも変化しうる.
平衡状態における水分量と配置は,これらのエネルギー変化が互いに拮抗する状態として決定される.
化学ポテンシャル$\mu$は定義上,その値が大きい場合,環境から水分を系内に取り込む際に要するエネルギーが大きい.
逆に$\mu$が小さく設定されていれば,わずからエネルギーを供給するだけで系内に水分を取り込むことができる.
これを物理的な状況にあてはめると,前者は粘土含水系が置かれた環境の湿度が低い場合に,
後者は湿度が高い場合に相当する.環境の湿度が低い場合,そこから所定の水分を粘土含水系に
移すために要するエネルギーは湿度が高い場合よりも大きいことは明らかであろう.
以上のことから,化学ポテンシャルの逆数$\mu^{-1}$には 粘土含水系が置かれた環境の湿度の
高低に相当する指標とみなすことができる.

CGMD法では水分量を表す変数は,水和水層の厚さ$s$であり水分子数$N$ではない.そこで,
CGMD法への実装において,水分量に関するエネルギー$U_N$の変化を
\begin{equation}
	dU_N=\tilde \mu ds, \ \ \left( \tilde \mu =\mu \frac{dN}{ds} \right)
	\label{eqn:}
\end{equation}
と表し,$\tilde \mu$を与えて計算を行う.$\tilde \mu$と$\mu$は定数倍だけの差であるため,
以下では$\tilde \mu$も化学ポテンシャルと呼ぶことにする.ただし,$\mu$がエネルギーの次元を持つのに対し,
$\tilde \mu$はエネルギー/長さの次元を持つことに注意する.
\subsection{水和エネルギー}
平衡状態においてどの程度の水分が系内に取り込まれるかは,化学ポテンシャル$\mu$の大小だけでなく,
$U_{LJ}$や$U_{hyd}$との兼ね合いで決まる.特に,水分量に関するエネルギー$U_N$と増減が反転する
水和エネルギー$U_{hyd}$の影響は大きい.また,水和エネルギーは,粘土鉱物や層間イオン種による
吸水や膨潤挙動の違いを生むという意味でも重要である.
従って,水和エネルギーと膨潤挙動の関係を理解することが,適切な組織構造モデルを得る
ために必要と言える.
%
\subsubsection{単調モデル}
図\ref{fig:fig1}(a)に,昨年度の研究で用いた水和エネルギーモデルを示す.ここでいう
水和エネルギーモデルとは,粗視化粒子の各方向に水和した,水和水の層厚と水和エネルギーの
関係を指す.図\ref{fig:fig1}(a)の水和エネルギーモデルは次の式で与えられる.
\begin{equation}
	u(s)=-u_{\infty} \left(1-e^{-\gamma s} \right)
	\label{eqn:u_s}
\end{equation}
ここに,$s$は水和水層の厚さを,$s_b$は
\begin{equation}
	\frac{\left| u(s_b) \right|}{u_{\infty}}=\frac{1}{2}
	\label{eqn:u_sb}
\end{equation}
となる基準距離を,$u_{\infty}$は$s\rightarrow \infty$(無限膨潤)
での水和エネルギーの大きさを表す.
全水和エネルギー$U_{hyd}$を与える式(\ref{eqn:Uhyd_tot})の右辺は,
式(\ref{eqn:u_s})をはじめとする水和エネルギーモデルによって計算する.
式(\ref{eqn:u_sb})より
\begin{equation}
	e^{-\gamma s_b}=\frac{1}{2} \ \ \Rightarrow \ \
	\gamma=\frac{\log 2}{s_b}
	\label{eqn:}
\end{equation}
だから,$s_b$を与えれば$\gamma$が決まる.
図\ref{fig:fig1}(b)は,$u(s)$の勾配$u'(s)$を示したもので,
\begin{equation}
	u'(s)=\frac{du}{ds}=-\gamma u_{\infty} e^{-\gamma s}
	\label{eqn:ud_s}
\end{equation}
より,
\begin{equation}
	\frac{u'(s_b)}{u'(0)}=\frac{1}{2}
	\label{eqn:ud_half}
\end{equation}
で,$s=s_b$において水和エネルギー関数の勾配も半減することが分かる.
このような水和エネルギーモデルを用いる理由は以下の通りである.
\begin{itemize}
\item
	単調減少する関数であること
\item
	$u(0)=0$かつ$s\rightarrow \infty$で有界であること
\item
	勾配の大きさが単調減少し,$s\rightarrow \infty$で0に収束すること
\end{itemize}
モンモリロナイトは水分子を強く水和し,最終的には分散状態に至る.
このことから,水和エネルギーは大局的に減少する関数でなければならない.
ただし,水和エネルギーが無限大となることは物理的でないため,2番目の条件を課す.
3番目の条件は,水和量が少ない程,より強く水分を吸着することを意味する.
式(\ref{eqn:u_s})は,これらの条件を満たすもののうち,単純かつ少数のパラメータで
規定できるものとして,昨年度採用したものである.以下ではこれを,単調モデルと呼ぶ.
なお,今回の計算では,基準距離を$s_b=0.1$nmとしている.
%
\subsubsection{振動モデル}
X線回折試験の結果から,Na型モンモリロナイトは相対湿度の増加に対して階段状に
層間距離を変化させることが知られている.層間距離は,概ね水分子整数個分程度となり,
水分子$n$個相当の層間距離にある状態を$n$層膨潤と呼ぶ.
Naモンモリロナイトは0層膨潤から1層,2層膨潤と推移し,例えば1.5層といった
中間的な膨潤状態を取ることは少ない.このような挙動は水和エネルギー関数に
単調モデルを用いた場合に生じ得ない.これは,整数次の膨潤状態に相当する
水和水層厚で,水和エネルギー関数が何ら特別な特徴を持たないためである.
そこで,水分子サイズに相当する周期で変化する振動成分を単調モデルに加えた
水和エネルギーモデルを用いることで,どのように膨潤挙動の違いが現れるかを調べる.

$n$層($n=1,2,\dots$)膨潤状態における水和水層の厚さ$s$は,
表\ref{tbl:tbl_sig}に示すように,0.15nmの倍数である.
そこで,周期0.15nmの振動成分を単調モデルに加えることを考える.
図\ref{fig:fig2}(a)に,黒の実線でこの方法で作成した振動成分を有する
水和エネルギーモデルを示す.以下,この水和エネルギーモデルを振動モデルと呼ぶ.
単調モデルに加えたで振動成分は,同図(b)に緑の実線で示したものである.
これは,周期$H=$0.15nmの三角波を平滑化したもので,元になる三角波は
周期毎に一定割合で減衰させている.
ここで,単調モデルを$u_0(s)$,これに加えた平滑化した減衰三角波成分を$\Delta u(s)$
とすれば,振動モデルは
\begin{equation}
	u(s)=u_0(s)+ \Delta u(s)
	\label{eqn:}
\end{equation}
と表される.単調モデルは,大きさを表す項$u_{\infty}$と,関数系を決める無次元の項
$\bar u_0$の積で,
\begin{equation}
	u_0(s)=u_{\infty} \bar{u}(s), \ \ (\bar{u}(s)=-(1-e^{-\gamma s}))
	\label{eqn:}
\end{equation}
と書くことができる.振動成分$\Delta u(s)$の詳細は,以下のようである.


はじめに,周期関数の基本区間を$[0,H)$とし,この区間における高さ1の単一の三角波を
\begin{equation}
	V_0(s)=\left\{
		\begin{array}{cc}
			s/H & (0\leq s < H) \\
			0 & ({\rm otherwise})
		\end{array}
	\right.
	\label{eqn:V0}
\end{equation}
と書く.これを平行移動して順次加えることで,$s\geq 0$における周期関数$V(s)$が次の
ように得られる.
\begin{equation}
	V(s)=\sum_{k=0}^{\infty} \beta ^{k} V_0(s-kH)
	\label{eqn:Vs}
\end{equation}
ここで,$\beta(<1)$は減衰率で,隣接する三角波の振幅比を与えるパラメータである.
$V(s)$は$s=kT$において不連続であるため,ガウス分布等を重み関数w(s)として移動平均
\begin{equation}
	\tilde V(s)=\int V(t)w(s-t)dt
	\label{eqn:mv_ave}
\end{equation}
で平滑化し,連続かつ滑らかな関数$\tilde V(s)$にする.
$V_0(s)$の平滑化を$\tilde V_0(s)$とすれば,平滑化された減衰三角波は
\begin{equation}
	\tilde V(s)=\sum_{k=0}^{\infty} \beta ^{k} \tilde V_0(s-kH)
	\label{eqn:Vst}
\end{equation}
と書ける.式(\ref{eqn:Vst})は無次元の関数なので,これを単調成分$u_0(s)$の
無次元項$\bar{u}(s)$と比率$\alpha$で加えるために,
$\Delta u(s)=u_\infty\alpha \tilde V(s)$とする.
以上より,振動モデルを与える式は次のように表される.
\begin{equation}
	u(s)=u_0(s)+ \alpha \tilde V(s)= u_{\infty}\left\{ 
	\bar{u}(s)+ \alpha \sum_{k=0}^{\infty} \beta ^{k} \tilde V_0(s-kH)
	\right\}
	\label{eqn:us_triwv}
\end{equation}
式(\ref{eqn:us_triwv})は3つのパラメータ$u_{\infty},\alpha$および$\beta$で
定義される.$u_{\infty}$は水和エネルギーの大きさを,$\alpha$は
単調成分と振動成分の比を,$\beta$は振動成分の減衰率をそれぞれ表す.
次節の計算例では,これらのパラメータは次のように与える.
\begin{equation}
	\alpha=0.22, \ \ \beta=0.6, \ \ H=0.15 \ \ [{\rm nm}]
	\label{eqn:}
\end{equation}
%--------------------
\begin{figure}[h]
	\begin{center}
	\includegraphics[width=1.0\linewidth]{Figs/fig1.eps} 
	\end{center}
	\caption{
		単調減少する水和エネルギーのモデル(単調モデル).
		$u_{\infty}$は$s\rightarrow \infty$での大きさ.$s_b$は$u=\frac{1}{2}u_{\infty}$となる
		水和層厚さを表す.
	} 
	\label{fig:fig1}
\end{figure}
%--------------------
\begin{figure}[h]
	\begin{center}
	\includegraphics[width=1.0\linewidth]{Figs/fig2.eps} 
	\end{center}
	\caption{
		水和エネルギーモデルとその成分(内訳).
		(a)振動モデル(i)と単調モデル(ii).
		(b)振動モデル(ii)の成分:単調(青)および振動成分(緑).
		赤は平滑化して振動成分を与えるために用いた三角波.
	} 
	\label{fig:fig2}
\end{figure}
%--------------------

\section{数値解析例}
本節では,系外から水分の出入りが生じるCGMD計算を,
化学ポテンシャル一定のものとで実施した結果を示す.
その際,あらたに考案した振動成分を含む水和エネルギーモデル(振動モデル)を用い,
どのような膨潤状態が現れるか調べる.以下では初期モデルの作成方法
を示し,次に,温度,体積,化学ポテンシャル一定で緩和計算を行った結果を示す.
最後に,温度,圧力,化学ポテンシャル一定での計算も可能であることを確認する
計算例を示し,吸水や排水によって体積膨張や収縮が生じることをみる.
なお,この計算例では,振動モデルとの比較も兼ね,昨年度用いた単調モデルでの
計算を行う.
\subsection{初期モデルの作成}
初期モデルの作成状況を図\ref{fig:fig3}に示す.
この図はユニットセル内に含まれる粘土分子の分布を示したものである.
(a)の図は,直線状の粘土分子をユニットセル内にランダムに配置した
もので,これを断熱圧縮して乾燥密度が約1.6g/cm$^{3}$となるモデルを作成する.
(b)はその結果を示し,(c)は断熱圧縮後の粘土分子配置を拡大したものである.
これらの計算では,水分量は2層膨潤状態($s$=0.3nm)に固定し,
粒子間の水分移動も,系内外での水分の授受も無いものとしている.
なお,断熱圧縮の過程では系の温度を制御していないため,
圧縮によって粒子系が得たエネルギーで,(b)の時点では非常に高い温度になっている.
そこで,(b)のモデルを300Kまで$250$psの間に冷却し,その後750ps一定温度
(300K)で緩和計算を行う.ここで,緩和とは,温度や体積などの外的条件を
固定し,系を平衡状態に向けて推移させることを意味する.
なお,冷却と緩和の間も水分量は2層膨潤状態で固定している.
図\ref{fig:fig3}(d)は,以上の結果として得られた状態を示す.
ここで(c)と(d)の図を見ると粘土分子の配置にはほとんど差がない.
これは,冷却によって粗視化粒子の振動は抑えられるが,密度が高く粒子位置は
ほとんど変化を起こすことができないためである.
\subsection{化学ポテンシャル一定での平衡化}
%	図4 :化学ポテンシャル一定での平衡化 図4
次に,図\ref{fig:fig3}(d)を新たな初期状態として,体積,温度,
化学ポテンシャル一定の条件で緩和計算を行い,別の平衡状態へ推移させる.
このとき,指定された化学ポテンシャルに応じて粘土含水系は吸排水し,
水分の総量や膨潤状態が変化する.
また,水和エネルギーモデルには振動モデルを用い,水和エネルギーの大きさ
を規定する式(\ref{eqn:us_triwv})の$u_{\infty}$は,緩和開始時の
相互作用ポテンシャル$\left. U_{LJ}\right|_{t=0}$に対して,
\begin{equation}
	u_{\infty}=5\frac{\left.U_{LJ}\right|_{t=0}}{N_p}
	\label{eqn:uinf_val}
\end{equation}
とした.ただし,$N_p$は粗視化粒子の数を表す.
$u_{\infty}$は無限膨潤状態での水和エネルギーの大きさだから,式(\ref{eqn:uinf_val})は,
水和エネルギーが最大で粒子あたりの相互作用エネルギーの5倍になるということを意味する.
一方,化学ポテンシャル$\tilde \mu$は単位長さあたりのエネルギーの次元を持つ.
そこで,
\begin{equation}
	\bar \mu = \frac{\tilde \mu }{\gamma u_{\infty}}
	\label{eqn:mu_bar}
\end{equation}
と無次元化する.$\gamma$は,単調成分(\ref{eqn:u_s})の減衰率で長さの逆数の次元
をもつ.また,規格化定数である$\gamma u_\infty$は,単調成分の
$s=0$での勾配という物理的意味をもつ.以下,化学ポテンシャルの大小は
式(\ref{eqn:mu_bar})の$\bar \mu$で表す.

図\ref{fig:fig4}は化学ポテンシャルによる影響をみるため,
$\bar{\mu}=0.1, 0.3$および$0.4$で1ns間,緩和計算を行った結果を
示したものである.この図の(a)は,緩和開始時の粘土分子の配置を,
(b)-(d)は各々の図に示した無次元化化学ポテンシャルに対する
緩和後の状態を示す.(b)と(c)のケースでは緩和開始時点から顕著な変化がみられない.
これに対して(d)では,は明らかに空隙が拡大している.化学ポテンシャルの逆数は,
は粘土含水系が置かれた環境の湿度指標考えてよく,大きな$\bar{\mu}$は低い湿度に相当する.
そのため(d)では顕著な排水が起き,結果として層外の空隙が拡がっている.
(a)や(b)と比べると,(c)でも若干の空隙の拡大は認められるが,
図\ref{fig:fig4}だけからは排水の結果であるか否かは判断し難い.
%	図5 :エネルギーの推移(化学ポテンシャル0.3の場合)
そこで,緩和途上でのエネルギーの変化を調べると図\ref{fig:fig5}のようになっている.
このグラフは,水和エネルギー$U_{hyd}$, 水分量に関するエネルギー$U_{N}$,
相互作用ポテンシャル$U_{LJ}$とそれらの和である全エネルギー$E$を示したものである.
このうち,系内に含まれる水分の増減は$U_{N}$の変化で示され,水分が増加(減少)するとき
$U_N$も増加(減少)する.
図\ref{fig:fig5}では,緩和の開始とともに$U_{N}$が減少して一定値に収束し,
当初よりも水分量が減少,すなわち排水が生じていることが分かる.
なお,水和エネルギーは最終的には当初より増加するが,
相互作用ポテンシャルは排水によって粒子間の相互作用が弱まり減少している.
以上の結果,全エネルギーは当初よりも小さな値に収束していることがこの図
に示されている.

%	図6 :水和水層厚のヒストグラム
% 山や谷の位置を参照するための記号を図中に書き入れること
次に,図\ref{fig:fig4}に示したそれぞれの状態が,どのような膨潤状態にあるか調べる.
そのために,水和水層の厚さ$s$の頻度分布を図\ref{fig:fig6}に示す.
水和水層の厚さ$s_i^\pm$は各粗視化粒子の上下方向で求められる.粗視化粒子数は
$N_p=$3,194個だから,図\ref{fig:fig6}は$N_p\times$2=6,388の
サンプルからなる頻度分布で,水和層の厚さに関する頻度(度数)を正規化等
行わずそのまま示している.
図\ref{fig:fig4}(a)の緩和開始時点では,全ての粒子の上下方向とも2層膨潤状態
に固定されている.そのため,水和水層の厚さは一律0.3nmで頻度分布はこの位置に
集中し(図\ref{fig:fig6}(a)),緩和終了時には指定された化学ポテンシャルに
応じて分布形が変化する.図\ref{fig:fig6}(b)の$\bar{\mu}=0.1$の場合,
ヒストグラムはほぼ一つの水和水層厚に集中した状態が維持されているが,
層厚は0.3から0.28nm程度に低下している.このことは,わずかに排水が生じ,
ほぼ単一の膨潤状態をとることを意味する.$\bar{\mu}=0.3$では,
同様な傾向の変化がよりはっきりと頻度分布に現れている.すなわち,
水和層の厚さがおよそ0.25nm程度まで低下し,分布幅が当初よりも大きくなるが,
依然として狭い範囲に集中している.
これに対して,組織構造(図\ref{fig:fig5}(d))に顕著な変化が現れる
$\bar{\mu}=0.4$のケースでは,2つのピーク2をもつ頻度分布となっている.
これは,排水により2つの膨潤状態に分離することを示し,
単調な水和エネルギーモデルを用いた場合には現れない挙動である.
なお,ピーク間の谷は水和エネルギー関数の山(極大点)付近に位置している.
つまり,図\ref{fig:fig6}(d)では, 水和エネルギーの山を避けるように
水和水層の分布が分離することを示している.さらに,2つのうち左側のピークは
右のものに比べて分布幅が狭い.これは,水和エネルギーの山から谷への勾配が
左側で大きく,右側で緩やかなことを反映したものである.
以上を踏まえて図\ref{fig:fig6}(b)および(c)の結果を見ると,
水和エネルギーの谷に近い(b)で狭い分布になっていたものが,(c)では排水によって
低い側へ谷からピークが離れることで分布幅が広がるとみることができる.
図\ref{fig:fig6}(d)において左のピーク幅が狭く,右のピーク幅が広いことは,
水和エネルギーの谷の深さに対応しており,この意味で(b),(c)の分布に
至る場合と同じメカニズムが(c)の結果でも作用していると言える.

%	図7 :組織構造とヒストグラム(初期水分分布をばらつかせた場合)
ここまでは,緩和開始時点で水分量が全ての粒子と全ての方向で2層膨潤状態にある
として計算を行ってきた.
%その結果,2層から1層膨潤状態付近へ移行し,
%ほとんどの場合水和層厚さの頻度分布は単峰的であった.
ここでは,初期状態で水分配置に大きなばらつきがある状態から温度,体積,化学
ポテンシャル一定の条件で緩和を行った結果を示す.初期モデルは,
図\ref{fig:fig7}(a)に示すような粘土分子配置のものを用いる.
このモデルは断熱圧縮から300Kへの冷却と緩和の過程で,系内での水分移動
を許すことで得られたものである.水分分布は図\ref{fig:fig7}(b)に示すように,
水和水の層厚が0から0.9nm程度の範囲で大きくばらついている.このことは
同図(a)の粘土分子配置において,粘土層間の間隙が非常に狭い箇所と
と広い箇所がある点にも見て取ることができる.
図\ref{fig:fig7}(c)と(d)は,$\bar{\mu}=0.1$で1ns間緩和計算を行った
結果を示したものである.(d)に示した水和層厚さの頻度分布は
1から3層膨潤状態付近に集中し,それより大きな膨潤状態は
ほとんど現れていない.なお,緩和過程では若干の吸水が生じているが
全水分量はほとんど変化していない.つまり,図\ref{fig:fig7}(d)は,
無水状態に近い箇所では吸水が,3層膨潤よりも水分が多い箇所では排水が生じ,
それ以外の箇所では1から3層膨潤のいずれかへ移行したことを示している.
このように特定の水和水層厚に膨潤状態が集中する挙動は,単調な水和エネルギーモデル
では生じず,水和エネルギーが振動成分をもつことによる効果を端的に示している.
\subsection{圧力一定での計算}
%	図8 :組織構造とヒストグラム(初期水分分布をばらつかせた場合)
最後に,圧力一定条件で行った緩和計算の結果について述べる.
これまでに示した体積一定条件での計算では,所定体積に系の状態を保つために膨潤圧が生じる.
これに対して圧力一定条件では,緩和の進行につれて所定の封圧のもとで体積膨張や収縮,すなわち
膨潤ひずみを生ずる.以下では,圧力一定の条件においても,指定された化学ポテンシャル
水和エネルギーを用いて緩和計算が可能であることを示す.
そのための初期モデルには,図\ref{fig:fig8}(a)のものを用いた.
このモデルは,図\ref{fig:fig3}(d)のものを,系内での水分移動を許す条件で
圧力50MPaで緩和させて得たものである.ただし,あまり極端な水分の偏りが生じないよう
最大水和水層の厚さは0.5nmに制限している.
このようにして用意した初期モデルを,温度300K, 圧力50MPaのもと, 指定した化学ポテンシャル
において緩和させる.図\ref{fig:fig8}の(b)-(d)はその結果得られた分子の配置を示したもので,
無次元化化学ポテンシャルは(b)から(d)の順に$\bar{\mu}=0.1,0.3$および0.4である.
なお,水和エネルギーには単調モデル(\ref{eqn:u_s})を用い,$u_\infty$や$s_b$の
設定はこれまでの計算と同じ値にしている.
ここで単調モデルを用いた理由は,圧力一定条件の緩和で生じる挙動が予見しやすいようにする
ためで,計算手法やプログラム上の制約によるものではない.
図\ref{fig:fig8}の(b)-(d)では,いずれも,初期状態からユニットセルのサイズ,すなわち
体積が変化している.(b)の結果では,粘土分子層間の距離が一均一化すると
ともに,層外の間隙が若干縮小し,ユニットセル自体はやや膨張している.
一方,化学ポテンシャルが大きい(c)と(d)のケースでは明らかな体積収縮が認められる.
これらのケースでも粘土層間の距離は均質化しているが,既存の層外の間隙はやや拡がり,
新たな層外間隙の生成も見られる.ただし,層外間隙の拡張や生成は,ユニットセル
自体の縮小を伴うため,体積一定で緩和したとき(図\ref{fig:fig4}(d))のような
顕著なものではない.
%	図9 :水和水層厚の頻度分布(TPmu計算,単調モデル,初期水分分布をばらつかせた場合)
これら図\ref{fig:fig8}の分子配置における,水和水層厚の頻度分布を図\ref{fig:fig9}に示す.
この図にあるように,緩和後の水和層厚分布は,どのケースでも鋭く,単峰的な分布となり,
一つの膨潤状態に集中している.これは,単調な水和エネルギーモデルを用いたことに起因し,
振動モデルを用いたときのような,特定の膨潤状態への選択的な移行や,複数の膨潤状態への分離は起きない.

以上のように,これまで継続的に開発を進めてきたCGMD法では,体積や圧力が一定のもとで吸排水が生じる
問題を扱うことができる.また,水和エネルギーモデルを変更することで,
膨潤挙動に変化が現れることから,層間イオンの種類や組成によって多様な膨潤挙動を
示すモンモリロナイトのモデル化を今後行うための手法として期待が持てる.
%--------------------
\begin{figure}[h]
	\begin{center}
	\includegraphics[width=0.9\linewidth]{Figs/fig3.eps} 
	\end{center}
	\caption{
		初期モデルの作成.(a)ランダムに配置された粘土分子.(b),(c) 断熱圧縮後の状態,(d)
		300Kへの冷却と温度一定で緩和後の状態.
	} 
	\label{fig:fig3}
\end{figure}
%--------------------
\begin{figure}[h]
	\begin{center}
	\includegraphics[width=0.9\linewidth]{Figs/fig4.eps} 
	\end{center}
	\caption{
		温度(300K),体積,化学ポテンシャル一定での緩和.
		(a)は緩和開始時の状態,(b),(c),(d)は各々の図に示された
		無次元化化学ポテンシャルで緩和後の状態.
	} 
	\label{fig:fig4}
\end{figure}
%--------------------
\begin{figure}[h]
	\begin{center}
	\includegraphics[width=0.7\linewidth]{Figs/fig5.eps} 
	\end{center}
	\caption{
		緩和過程におけるエネルギーの推移($\bar{\mu}=0.3の場合$).
	} 
	\label{fig:fig5}
\end{figure}
%--------------------
\begin{figure}[h]
	\begin{center}
	\includegraphics[width=0.9\linewidth]{Figs/fig6.eps} 
	\end{center}
	\caption{
		水和層厚さの頻度分布.(a)は緩和開始時,(b),(c),(d)は
		各々の図に示された無次元化化学ポテンシャルでの緩和後.
	} 
	\label{fig:fig6}
\end{figure}
%--------------------
\begin{figure}[h]
	\begin{center}
	\includegraphics[width=1.0\linewidth]{Figs/fig7.eps} 
	\end{center}
	\caption{
		化学ポテンシャル一定条件での緩和による粘土分子配置と水和層厚分布の変化.
		無次元化化学ポテンシャル$\bar{\mu}=0.1$,初期状態で水分配置に大きなばら
		つきがある場合.
	} 
	\label{fig:fig7}
\end{figure}
%--------------------
\begin{figure}[h]
	\begin{center}
	\includegraphics[width=0.9\linewidth]{Figs/fig8.eps} 
	\end{center}
	\caption{
		温度(300K),圧力(50MPa),化学ポテンシャル一定での緩和.
		(a)は緩和開始時の状態,(b),(c),(d)は各々の図に示された
		無次元化化学ポテンシャルで緩和後の状態.
		水和エネルギーに単調モデルを用いた場合の結果.
	} 
	\label{fig:fig8}
\end{figure}
%--------------------
\begin{figure}[h]
	\begin{center}
	\includegraphics[width=0.9\linewidth]{Figs/fig9.eps} 
	\end{center}
	\caption{
		水和層厚さの頻度分布.(a)は緩和開始時,(b),(c),(d)は
		各々の図に示された無次元化化学ポテンシャルでの緩和後.
		それぞれの図は,図\ref{fig:fig8}に示した粘土分子配置
		に対応するもの.
	} 
	\label{fig:fig9}
\end{figure}
%--------------------

%\input{4_results.tex シミュレーション結果}
\section{まとめと今後の課題}
\hspace{\parindent}
本年度の研究では、粗視化分子動力学法(CGMD法)により、温度、化学ポテンシャルに
加え、体積あるいは圧力を指定した粘土含水系の凝集や緩和のシミュレーションが可能
であることを示した.化学ポテンシャルは、物理的には粘土含水系が置かれた環境の湿度に
相当する指標と理解することができるため、これにより、体積一定の元で膨潤圧の計算を、
圧力一定の元では膨潤量の計算を行うことが可能となった。
さらに、粘土分子表面への水和挙動を特徴付ける水和エネルギーモデルに、昨年度
までの単調な関数だけでなく、振動成分を加えたより複雑なモデルを用いた解析を試行した。
その結果、水和エネルギーの振動成分を加えることで、 特定の膨潤を選択する
メカニズムが働くことが示された。より正確には、水和エネルギーの谷の深い部分では
膨潤状態は狭い範囲に集中し、浅い部分ではばらつきが生じる。さらに、水和エネルギーの
山の部分を避けるように膨潤状態が選択され、その結果、離散的な複数の膨潤状態に分離する.

今後は、水和エネルギー関数と膨潤挙動の関係をより詳細に調べ、
水和エネルギーモデルを精緻化することが課題となる。
その際、シミュレーション結果と実験によって得られた膨潤曲線やX線回折パターンとの
比較や、全原子分子動力学計算による水和挙動の評価と、そこで得られた知見をCGMD法に
反映することが必要と考えられる。
これらの取り組みを経て現実的な水和エネルギーモデルを見出すことができれば
層間イオン種や組成、核種移行状況に応じた膨潤や組織形成のシミュレーションが可能となり、
緩衝材の設計や性能評価にも有用な知見を提供することができるようになると期待できる。
%
%\begin{thebibliography}{99}
%\bibitem{Warren}
%	B.E. Warren, "X-ray diffraction", Dover Publication, New York, 1990.
%\bibitem{Koide}
%	小出昭一郎,"物理現象のフーリエ解析", ちくま学芸文庫, 2018.
%\end{thebibliography}
\end{document}
%%%%%%%%%%%%%%%%%%%%%%%%%%%%%%%%%%%%%%%%%%%%%%%%%%%%%%

