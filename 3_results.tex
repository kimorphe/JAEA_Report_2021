\section{数値解析例}
本節では,系外から水分の出入りが生じるCGMD計算を,
化学ポテンシャル一定のものとで実施した結果を示す.
その際,あらたに考案した振動成分を含む水和エネルギーモデル(振動モデル)を用い,
どのような膨潤状態が現れるか調べる.以下では初期モデルの作成方法
を示し,次に,温度,体積,化学ポテンシャル一定で緩和計算を行った結果を示す.
最後に,温度,圧力,化学ポテンシャル一定での計算も可能であることを確認する
計算例を示し,吸水や排水によって体積膨張や収縮が生じることをみる.
なお,この計算例では,振動モデルとの比較も兼ね,昨年度用いた単調モデルでの
計算を行う.
\subsection{初期モデルの作成}
初期モデルの作成状況を図\ref{fig:fig3}に示す.
この図はユニットセル内に含まれる粘土分子の分布を示したものである.
(a)の図は,直線状の粘土分子をユニットセル内にランダムに配置した
もので,これを断熱圧縮して乾燥密度が約1.6g/cm$^{3}$となるモデルを作成する.
(b)はその結果を示し,(c)は断熱圧縮後の粘土分子配置を拡大したものである.
これらの計算では,水分量は2層膨潤状態($s$=0.3nm)に固定し,
粒子間の水分移動も,系内外での水分の授受も無いものとしている.
なお,断熱圧縮の過程では系の温度を制御していないため,
圧縮によって粒子系が得たエネルギーで,(b)の時点では非常に高い温度になっている.
そこで,(b)のモデルを300Kまで$250$psの間に冷却し,その後750ps一定温度
(300K)で緩和計算を行う.ここで,緩和とは,温度や体積などの外的条件を
固定し,系を平衡状態に向けて推移させることを意味する.
なお,冷却と緩和の間も水分量は2層膨潤状態で固定している.
図\ref{fig:fig3}(d)は,以上の結果として得られた状態を示す.
ここで(c)と(d)の図を見ると粘土分子の配置にはほとんど差がない.
これは,冷却によって粗視化粒子の振動は抑えられるが,密度が高く粒子位置は
ほとんど変化を起こすことができないためである.
\subsection{化学ポテンシャル一定での平衡化}
%	図4 :化学ポテンシャル一定での平衡化 図4
次に,図\ref{fig:fig3}(d)を新たな初期状態として,体積,温度,
化学ポテンシャル一定の条件で緩和計算を行い,別の平衡状態へ推移させる.
このとき,指定された化学ポテンシャルに応じて粘土含水系は吸排水し,
水分の総量や膨潤状態が変化する.
また,水和エネルギーモデルには振動モデルを用い,水和エネルギーの大きさ
を規定する式(\ref{eqn:us_triwv})の$u_{\infty}$は,緩和開始時の
相互作用ポテンシャル$\left. U_{LJ}\right|_{t=0}$に対して,
\begin{equation}
	u_{\infty}=5\frac{\left.U_{LJ}\right|_{t=0}}{N_p}
	\label{eqn:uinf_val}
\end{equation}
とした.ただし,$N_p$は粗視化粒子の数を表す.
$u_{\infty}$は無限膨潤状態での水和エネルギーの大きさだから,式(\ref{eqn:uinf_val})は,
水和エネルギーが最大で粒子あたりの相互作用エネルギーの5倍になるということを意味する.
一方,化学ポテンシャル$\tilde \mu$は単位長さあたりのエネルギーの次元を持つ.
そこで,
\begin{equation}
	\bar \mu = \frac{\tilde \mu }{\gamma u_{\infty}}
	\label{eqn:mu_bar}
\end{equation}
と無次元化する.$\gamma$は,単調成分(\ref{eqn:u_s})の減衰率で長さの逆数の次元
をもつ.また,規格化定数である$\gamma u_\infty$は,単調成分の
$s=0$での勾配という物理的意味をもつ.以下,化学ポテンシャルの大小は
式(\ref{eqn:mu_bar})の$\bar \mu$で表す.

図\ref{fig:fig4}は化学ポテンシャルによる影響をみるため,
$\bar{\mu}=0.1, 0.3$および$0.4$で1ns間,緩和計算を行った結果を
示したものである.この図の(a)は,緩和開始時の粘土分子の配置を,
(b)-(d)は各々の図に示した無次元化化学ポテンシャルに対する
緩和後の状態を示す.(b)と(c)のケースでは緩和開始時点から顕著な変化がみられない.
これに対して(d)では,は明らかに空隙が拡大している.化学ポテンシャルの逆数は,
は粘土含水系が置かれた環境の湿度指標考えてよく,大きな$\bar{\mu}$は低い湿度に相当する.
そのため(d)では顕著な排水が起き,結果として層外の空隙が拡がっている.
(a)や(b)と比べると,(c)でも若干の空隙の拡大は認められるが,
図\ref{fig:fig4}だけからは排水の結果であるか否かは判断し難い.
%	図5 :エネルギーの推移(化学ポテンシャル0.3の場合)
そこで,緩和途上でのエネルギーの変化を調べると図\ref{fig:fig5}のようになっている.
このグラフは,水和エネルギー$U_{hyd}$, 水分量に関するエネルギー$U_{N}$,
相互作用ポテンシャル$U_{LJ}$とそれらの和である全エネルギー$E$を示したものである.
このうち,系内に含まれる水分の増減は$U_{N}$の変化で示され,水分が増加(減少)するとき
$U_N$も増加(減少)する.
図\ref{fig:fig5}では,緩和の開始とともに$U_{N}$が減少して一定値に収束し,
当初よりも水分量が減少,すなわち排水が生じていることが分かる.
なお,水和エネルギーは最終的には当初より増加するが,
相互作用ポテンシャルは排水によって粒子間の相互作用が弱まり減少している.
以上の結果,全エネルギーは当初よりも小さな値に収束していることがこの図
に示されている.

%	図6 :水和水層厚のヒストグラム
% 山や谷の位置を参照するための記号を図中に書き入れること
次に,図\ref{fig:fig4}に示したそれぞれの状態が、どのような膨潤状態にあるか調べる.
そのために,水和水層の厚さ$s$の頻度分布を図\ref{fig:fig6}に示す.
水和水層の厚さ$s_i^\pm$は各粗視化粒子の上下方向で求められる.粗視化粒子数は
$N_p=$3,194個だから、図\ref{fig:fig6}は$N_p\times$2=6,388の
サンプルからなる頻度分布で、水和層の厚さに関する頻度(度数)を正規化等
行わずそのまま示している.
図\ref{fig:fig4}(a)の緩和開始時点では,全ての粒子の上下方向とも2層膨潤状態
に固定されている.そのため,水和水層の厚さは一律0.3nmで頻度分布はこの位置に
集中し(図\ref{fig:fig6}(a)),緩和終了時には指定された化学ポテンシャルに
応じて分布形が変化する.図\ref{fig:fig6}(b)の$\bar{\mu}=0.1$の場合,
ヒストグラムはほぼ一つの水和水層厚に集中した状態が維持されているが,
層厚は0.3から0.28nm程度に低下している。このことは、わずかに排水が生じ,
ほぼ単一の膨潤状態をとることを意味する。$\bar{\mu}=0.3$では、
同様な傾向の変化がよりはっきりと頻度分布に現れている.すなわち,
水和層の厚さがおよそ0.25nm程度まで低下し、分布幅が当初よりも大きくなるが、
依然として狭い範囲に集中している.
これに対して、組織構造(図\ref{fig:fig5}(d))に顕著な変化が現れる
$\bar{\mu}=0.4$のケースでは、2つのピーク2をもつ頻度分布となっている.
これは,排水により2つの膨潤状態に分離することを示し,
単調な水和エネルギーモデルを用いた場合には現れない挙動である。
なお,ピーク間の谷は水和エネルギー関数の山(極大点)付近に位置している.
つまり,図\ref{fig:fig6}(d)では, 水和エネルギーの山を避けるように
水和水層の分布が分離することを示している.さらに,2つのうち左側のピークは
右のものに比べて分布幅が狭い。これは,水和エネルギーの山から谷への勾配が
左側で大きく,右側で緩やかなことを反映したものである.
以上を踏まえて図\ref{fig:fig6}(b)および(c)の結果を見ると,
水和エネルギーの谷に近い(b)で狭い分布になっていたものが,(c)では排水によって
低い側へ谷からピークが離れることで分布幅が広がるとみることができる。
図\ref{fig:fig6}(d)において左のピーク幅が狭く、右のピーク幅が広いことは、
水和エネルギーの谷の深さに対応しており、この意味で(b),(c)の分布に
至る場合と同じメカニズムが作用しているということができる。

%	図7 :組織構造とヒストグラム(初期水分分布をばらつかせた場合)
ここまで,平衡化開始時の水分量が全ての粒子,全ての方向で2層膨潤状態にある
として計算を行ってきた.そのため,排水によって2層から1層膨潤状態付近へ移行し,
ほとんどの場合水和層厚さの頻度分布は単峰的であった.そこで,初期状態で水分
配置に大きなばらつきがある場合について,温度,体積,化学ポテンシャル一定の
条件で緩和計算を行う.初期モデルは,図\ref{fig:fig7}(a)に示すような粘土分子配置
のものを用いる.このモデルは,断熱圧縮から300Kでの緩和の途上で,
系内での水分移動を許すことで得られたものである.
水分分布は図\ref{fig:fig7}(b)にあるように,水和水層厚が
0から0.9nm程度の範囲で大きくばらついている.このことは,同図(a)の粘土分子
の配置において,粘土層間の間隙が場所によって非常に狭いところ
と広いところがある点にも見て取ることができる.
図\ref{fig:figr7}(c)は,$\bar{\mu}=0.1$で平衡化した結果を,
(d)はそのときの水和水層厚さの頻度分布を示している.頻度分布は
1,2および3層膨潤状態付近に集中し,それより大きな膨潤状態をとる
粒子はほとんど無い.なお,緩和過程では若干の吸水が起きているが,
全水分量には緩和前後でほとんど変化していない.つまり,
水分の配置のみが変化し,無水状態に近い箇所では吸水が,
3層膨潤よりも水分が多い箇所では排水がおき,
それ以外の箇所では1から3層膨潤に近い状態に移行することを示している.
このように,特定の水和水層厚に膨潤状態が集中する挙動は単調な
水和エネルギーモデルでは生じず,水和エネルギーが振動成分をもつことによる
効果を示している.
\subsection{圧力一定での計算}
%	図8 :組織構造とヒストグラム(初期水分分布をばらつかせた場合)
最後に,圧力一定条件で行った平衡化の結果を示す.
体積一定条件での緩和では,所定の体積に系を保つために生じる膨潤圧が計算できることに対し,
圧力一定条件での緩和は膨潤ひずみが発生し,所定の封圧のもとで生じる膨潤量を得ることができる.
以下,圧力一定条件において,指定された化学ポテンシャルでの緩和計算が可能であることを示す.
緩和計算の初期モデルには,図\ref{fig:fig8}(a)のものを用いた.
このモデルは,図\ref{fig:fig3}(d)に示したモデルを,
系内での水分移動を許す条件で圧力50MPaでした結果である.
そのため水分分布は系内で変化し,その様子は図\ref{fig:fig9}(a)のようである.
ただしここでは,極端な水分の偏りが生じないよう,最大水和水層の厚さを0.5nmに
制限して計算を行った.
このようにして用意した初期モデルを,温度300K, 圧力50MPaのもと,
指定した化学ポテンシャルにおいて平衡化する.
図\ref{fig:fig8}の(b)-(d)はその結果得られた分子配置を示したもので,
無次元化化学ポテンシャルは順に$\bar{\mu}=0.1,0.3$および0.4である.
なお,水和エネルギーには単調モデルを用いて計算を行っている.
これは,圧力一定条件で生じる挙動が予見しやすいように,単純な水和エネルギーモデル
を採用したことによるもので,計算手法やプログラム上の制約ではない.
図\ref{fig:fig8}(b)-(d)ではいずれも,初期状態からユニットセルのサイズが変化,
すなわち体積変化が生じている.(b)ではわずかな吸水膨張が見られ,粘土分子層間の
距離が一様に,層外の間隙がやや縮小している様子が見られる.
一方,化学ポテンシャルが相対的に大きい(c)と(d)の場合には,明らかな体積の
収縮が認められる.これらのケースでも粘土層間の距離は均質化しているが,層外
の間隙はやや拡がり,新たな層外間隙の生成も見受けられる.ただし,体積の収縮が
あるため全体として体積一定で緩和したとき(図\ref{fig:fig4}(d))のような
顕著な層外間隙の生成は生じていない.
%	図9 :水和水層厚の頻度分布(TPmu計算,単調モデル,初期水分分布をばらつかせた場合)
膨潤状態に関して言えば,図\ref{fig:fig9}に示すように,どのケースでも
鋭い,単峰的な分布となっており一つの膨潤状態に集中する.これは
明らかに,単調な水和エネルギーモデルを用いたことによる影響で,
特定の膨潤状態へ選択的に移行することや
複数の膨潤状態へ分離することがない.以上より,これまで継続的に開発を
進めてきたCGMD法では,体積あるいは圧力一定のもとで吸排水が生じる
問題を扱うことが可能であることが確認できる.また,水和エネルギー
モデルの選択によって異なる膨潤挙動が現れる.これは,層間イオンの
種類や組成によって多様な膨潤挙動を示すモンモリロナイトのモデル化を
行うための手法としての利用が期待できる.
%--------------------
\begin{figure}[h]
	\begin{center}
	\includegraphics[width=0.9\linewidth]{Figs/fig3.eps} 
	\end{center}
	\caption{
		初期モデルの作成.(a)ランダムに配置された粘土分子.(b),(c) 断熱圧縮後の状態,(d)
		300Kへの冷却と温度一定での平衡化結果.
	} 
	\label{fig:fig3}
\end{figure}
%--------------------
\begin{figure}[h]
	\begin{center}
	\includegraphics[width=0.9\linewidth]{Figs/fig4.eps} 
	\end{center}
	\caption{
		温度(300K),体積,化学ポテンシャル一定での平衡化.
		(a)は平衡化前の状態,(b),(c),(d)は各々の図に示された
		無次元化化学ポテンシャルで平衡化後の状態.
	} 
	\label{fig:fig4}
\end{figure}
%--------------------
\begin{figure}[h]
	\begin{center}
	\includegraphics[width=0.7\linewidth]{Figs/fig5.eps} 
	\end{center}
	\caption{
		平衡化の過程におけるエネルギーの推移($\bar{\mu}=0.3の場合$).
	} 
	\label{fig:fig5}
\end{figure}
%--------------------
\begin{figure}[h]
	\begin{center}
	\includegraphics[width=0.9\linewidth]{Figs/fig6.eps} 
	\end{center}
	\caption{
		水和水層厚の頻度分布.(a)は平衡化前,(b),(c),(d)は
		各々の図に示された無次元化化学ポテンシャルでの平衡化後.
	} 
	\label{fig:fig6}
\end{figure}
%--------------------
\begin{figure}[h]
	\begin{center}
	\includegraphics[width=1.0\linewidth]{Figs/fig7.eps} 
	\end{center}
	\caption{
		化学ポテンシャル一定条件での緩和による,粘土分子配置と水和水層厚分布の変化.
		無次元化化学ポテンシャル$\bar{\mu}=0.1$,初期状態で水分配置に大きなばら
		つきがある場合.
	} 
	\label{fig:fig7}
\end{figure}
%--------------------
\begin{figure}[h]
	\begin{center}
	\includegraphics[width=0.9\linewidth]{Figs/fig8.eps} 
	\end{center}
	\caption{
		温度(300K),圧力(50MPa),化学ポテンシャル一定での平衡化.
		(a)は平衡化前の状態,(b),(c),(d)は各々の図に示された
		無次元化化学ポテンシャルで平衡化後の状態.
		水和エネルギーに単調モデルを用いた場合の結果.
	} 
	\label{fig:fig8}
\end{figure}
%--------------------
\begin{figure}[h]
	\begin{center}
	\includegraphics[width=0.9\linewidth]{Figs/fig9.eps} 
	\end{center}
	\caption{
		水和水層厚の頻度分布.(a)は平衡化前,(b),(c),(d)は
		各々の図に示された無次元化化学ポテンシャルでの平衡化後.
		それぞれの図は,図\ref{fig:fig8}に示した粘土分子配置
		に対応するもの.
	} 
	\label{fig:fig9}
\end{figure}
%--------------------
